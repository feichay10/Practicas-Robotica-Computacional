\documentclass[11pt]{report}

% Paquetes y configuraciones adicionales
\usepackage{graphicx}
\usepackage[export]{adjustbox}
\usepackage{caption}
\usepackage{float}
\usepackage{titlesec}
\usepackage{geometry}
\usepackage[hidelinks]{hyperref}
\usepackage{titling}
\usepackage{titlesec}
\usepackage{parskip}
\usepackage{wasysym}
\usepackage{tikzsymbols}
\usepackage{fancyvrb}
\usepackage{xurl}
\usepackage{hyperref}
\usepackage{subcaption}

\usepackage{listings}
\usepackage{xcolor}

\usepackage[spanish]{babel}

\newcommand{\subtitle}[1]{
  \posttitle{
    \par\end{center}
    \begin{center}\large#1\end{center}
    \vskip0.5em}
}

% Configura los márgenes
\geometry{
  left=2cm,   % Ajusta este valor al margen izquierdo deseado
  right=2cm,  % Ajusta este valor al margen derecho deseado
  top=3cm,
  bottom=3cm,
}

% Configuración de los títulos de las secciones
\titlespacing{\section}{0pt}{\parskip}{\parskip}
\titlespacing{\subsection}{0pt}{\parskip}{\parskip}
\titlespacing{\subsubsection}{0pt}{\parskip}{\parskip}

% Redefinir el formato de los capítulos y añadir un punto después del número
\makeatletter
\renewcommand{\@makechapterhead}[1]{%
  \vspace*{0\p@} % Ajusta este valor para el espaciado deseado antes del título del capítulo
  {\parindent \z@ \raggedright \normalfont
    \ifnum \c@secnumdepth >\m@ne
        \huge\bfseries \thechapter.\ % Añade un punto después del número
    \fi
    \interlinepenalty\@M
    #1\par\nobreak
    \vspace{10pt} % Ajusta este valor para el espacio deseado después del título del capítulo
  }}
\makeatother

% Configura para que cada \chapter no comience en una pagina nueva
\makeatletter
\renewcommand\chapter{\@startsection{chapter}{0}{\z@}%
    {-3.5ex \@plus -1ex \@minus -.2ex}%
    {2.3ex \@plus.2ex}%
    {\normalfont\Large\bfseries}}
\makeatother

% Configurar los colores para el código
\definecolor{codegreen}{rgb}{0,0.6,0}
\definecolor{codegray}{rgb}{0.5,0.5,0.5}
\definecolor{codepurple}{rgb}{0.58,0,0.82}
\definecolor{backcolour}{rgb}{0.95,0.95,0.92}

% Configurar el estilo para el código
\lstdefinestyle{mystyle}{
  backgroundcolor=\color{backcolour},   
  commentstyle=\color{codegreen},
  keywordstyle=\color{magenta},
  numberstyle=\tiny\color{codegray},
  stringstyle=\color{codepurple},
  basicstyle=\ttfamily\footnotesize,
  breakatwhitespace=false,         
  breaklines=true,                 
  captionpos=b,                    
  keepspaces=true,                 
  numbers=left,                    
  numbersep=5pt,                  
  showspaces=false,                
  showstringspaces=false,
  showtabs=false,                  
  tabsize=2
}

%==============================================================================
% Cosas para la documentación LateX
% % Sangría
% \setlength{\parindent}{1em}Texto

% % Quitar sangría
% \noindent

% % Punto
% \CIRCLE \ \ \textbf{Texto} \emph{algo}
% \begin{itemize}
%   \item \textbf{Negrita:} Texto
%   \item \textbf{Negrita:} Texto
% \end{itemize}

% % Introducir código
% \begin{center}
%   \begin{BVerbatim}
%     ... Código
%   \end{BVerbatim}
% \end{center}

% Poner una imagen
% \begin{figure}[H]
%   \centering
%   \includegraphics[scale=0.55]{img/}
%   \caption{Exportación de la base de datos en formato sql}
%   \label{fig:exportación de la base de datos en formato sql}
% \end{figure}

% Poner dos imágenes
% \begin{figure}[H]
%   \begin{subfigure}{0.5\textwidth}
%     \centering
%     \includegraphics[scale=0.45]{img/}
%     \caption{Texto imagen 1}
%   \end{subfigure}%
%   \begin{subfigure}{0.5\textwidth}
%     \centering
%     \includegraphics[scale=0.45]{img/}
%     \caption{Texto imagen 2}
%   \end{subfigure}
%   \caption{Texto general}
% \end{figure}

% % Poner una tabla
% \begin{table}[H]
%   \centering
%   \begin{tabular}{|c|c|c|c|}
%     \hline
%     \textbf{Campo 1} & \textbf{Campo 2} & \textbf{Campo 3} & \textbf{Campo 4} \\ \hline
%     Texto & Texto & Texto & Texto \\ \hline
%     Texto & Texto & Texto & Texto \\ \hline
%     Texto & Texto & Texto & Texto \\ \hline
%     Texto & Texto & Texto & Texto \\ \hline
%   \end{tabular}
%   \caption{Nombre de la tabla}
%   \label{tab:nombre de la tabla}
% \end{table}

% % Poner codigo de un lenguaje a partir de un archivo
% \lstset{style=mystyle}
% The next code will be directly imported from a file
% \lstinputlisting[language=Python]{code.py}

% “Texto entre comillas dobles”

%==============================================================================

\begin{document}

% Portada del informe
\title{Informe de prácticas}
\subtitle{Robótica Computacional}
\author{Cheuk Kelly Ng Pante (alu0101364544@ull.edu.es)}
\date{\today}

\maketitle

\pagestyle{empty} % Desactiva la numeración de página para el índice

% Índice
\tableofcontents

% Nueva página
\cleardoublepage

\pagestyle{plain} % Vuelve a activar la numeración de página
\setcounter{page}{1} % Reinicia el contador de página a 1

% Secciones del informe
% Capitulo 1
\chapter{Cinematica Directa}
La cinemática directa es una rama de la robótica y la mecánica que se ocupa de la relación entre los movimientos
de los eslabones de un robot y las variables que los controlan. En otras palabras, la cinemática directa es el
problema de encontrar la posición y orientación del extremo del robot, dado el conjunto de parámetros que
definen las posiciones y orientaciones de todos los eslabones.

Para explicar la cinemática directa, se utilizará el sistema de coordenadas de Denavit-Hartenberg (DH). El sistema
de coordenadas de Denavit-Hartenberg es un sistema de coordenadas utilizado para modelar cinemática directa e
inversa de robots articulados. El sistema de coordenadas de Denavit-Hartenberg se basa en cuatro parámetros
asociados a cada articulación. Estos parámetros son:
\begin{itemize}
  \item $d_i$: Distancia entre los ejes $z_{i-1}$ y $z_i$ a lo largo del eje $x_i$.
  \item $\theta_i$: Ángulo entre los ejes $z_{i-1}$ y $z_i$ alrededor del eje $x_i$.
  \item $a_i$: Distancia entre los ejes $x_{i-1}$ y $x_i$ a lo largo del eje $z_{i-1}$.
  \item $\alpha_i$: Ángulo entre los ejes $x_{i-1}$ y $x_i$ alrededor del eje $z_{i-1}$.
\end{itemize}

Los parámetros DH se pueden calcular segun la tabla siguiente:
\begin{table}[H]
  \centering
  \begin{tabular}{|c|c|c|c|}
    \hline
    \textbf{}           & \textbf{A}                 & \textbf{B}                 & \textbf{C}         \\ \hline
    \textbf{$d_i$}      & \texttt{$O_{i-1}$}         & \texttt{$Z_{i-1}\cap X_i$} & \texttt{$Z_{i-1}$} \\ \hline
    \textbf{$\theta_i$} & \texttt{$X_{i-1}$}         & \texttt{$X_{i}$}           & \texttt{$Z_{i-1}$} \\ \hline
    \textbf{$a_i$}      & \texttt{$Z_{i-1}\cap X_i$} & \texttt{$O_{i}$}           & \texttt{$X_{i}$}   \\ \hline
    \textbf{$\alpha_i$} & \texttt{$Z_{i-1}$}         & \texttt{$Z_{i}$}           & \texttt{$X_{i}$}   \\ \hline
  \end{tabular}
  \caption{Parámetros DH}
  \label{tab:parametros dh}
\end{table}

Para la explicación de la cinemática directa, se utilizará el manipulador 3:
\begin{figure}[H]
  \centering
  \includegraphics[scale=0.24]{img/manipulador.png}
  \caption{Manipulador ejemplo}
  \label{fig:manipulador 3}
\end{figure}

Para calcular la cinemática directa, primero vamos a calcular los parámetros DH:
\begin{figure}[H]
  \centering
  \includegraphics[scale=0.4]{img/parametros_dh.png}
  \caption{Parámetros DH del manipulador 3} 
  \label{fig:manipulador 3 dh}
\end{figure}

Como se puede observar en la figura \ref{fig:manipulador 3 dh}, antes de calcular los parámetros DH, se ha
asignado unas variables de entrada, estas corresponden a los ángulos de las articulaciones del manipulador
y estas son los valores que se introducen al ejecutar el programa. Por ejemplo, si se introduce los valores
5 0 5 90 45, el resultado de la cinemática directa sería la siguiente:
\begin{figure}[H]
  \centering
  \includegraphics[scale=0.6]{img/cinematica_directa.png}
  \caption{Cinemática directa del manipulador 3} 
  \label{fig:manipulador 3 cinematica directa}
\end{figure}

\newpage

Después, se calculado la matriz de transformación de cada articulación para este manipulador:
\begin{figure}[H]
  \centering
  \includegraphics[scale=0.5]{img/matriz_transformacion.png}
  \caption{Matriz de transformación del manipulador 3}
  \label{fig:manipulador 3 matriz transformacion}
\end{figure}

Luego, la transformación de cada articulación, especificando los datos de la tabla de Denavit Hartenberg especificadas en la figura \ref{fig:manipulador 3 dh}:
\begin{figure}[H]
  \centering
  \includegraphics[scale=0.5]{img/transformacion_articulacion.png}
  \caption{Transformación de cada articulación del manipulador 3}
  \label{fig:manipulador 3 transformacion articulacion}
\end{figure}

Y por último, el resultado de la cinemática directa del manipulador 3:
\begin{figure}[H]
  \centering
  \includegraphics[scale=0.5]{img/resultado_cd.png}
  \caption{Cinemática directa del manipulador 3}
\end{figure}

\end{document}