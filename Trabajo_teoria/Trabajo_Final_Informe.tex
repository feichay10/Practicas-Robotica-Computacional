\documentclass[11pt]{report}

% Paquetes y configuraciones adicionales
\usepackage{amsmath, amsthm, amssymb} % Paquetes matemáticos
\usepackage[utf8]{inputenc} % Codificación .tex
\usepackage[T1]{fontenc} % Codificación .pdf
\usepackage{graphicx}
\usepackage[export]{adjustbox}
\usepackage{caption}
\usepackage{float}
\usepackage{titlesec}
\usepackage{geometry}
\usepackage[hidelinks]{hyperref}
\usepackage{titling}
\usepackage{titlesec}
\usepackage{parskip}
\usepackage{wasysym}
\usepackage{tikzsymbols}
\usepackage{fancyvrb}
\usepackage{xurl}
\usepackage{hyperref}
\usepackage{subcaption}
\usepackage{listings}
\usepackage{xcolor}
\usepackage[spanish]{babel}

\newcommand{\subtitle}[1]{
  \posttitle{
    \par\end{center}
    \begin{center}\large#1\end{center}
    \vskip0.5em}
}

% Configura los márgenes
\geometry{
  left=2cm,   % Ajusta este valor al margen izquierdo deseado
  right=2cm,  % Ajusta este valor al margen derecho deseado
  top=3cm,
  bottom=3cm,
}

% Configuración de los títulos de las secciones
\titlespacing{\section}{0pt}{\parskip}{\parskip}
\titlespacing{\subsection}{0pt}{\parskip}{\parskip}
\titlespacing{\subsubsection}{0pt}{\parskip}{\parskip}

% Redefinir el formato de los capítulos y añadir un punto después del número
\makeatletter
\renewcommand{\@makechapterhead}[1]{%
  \vspace*{0\p@} % Ajusta este valor para el espaciado deseado antes del título del capítulo
  {\parindent \z@ \raggedright \normalfont
    \ifnum \c@secnumdepth >\m@ne
        \huge\bfseries \thechapter.\ % Añade un punto después del número
    \fi
    \interlinepenalty\@M
    #1\par\nobreak
    \vspace{10pt} % Ajusta este valor para el espacio deseado después del título del capítulo
  }}
\makeatother

% Configura para que cada \chapter no comience en una pagina nueva
\makeatletter
\renewcommand\chapter{\@startsection{chapter}{0}{\z@}%
    {-3.5ex \@plus -1ex \@minus -.2ex}%
    {2.3ex \@plus.2ex}%
    {\normalfont\Large\bfseries}}
\makeatother

% Configurar los colores para el código
\definecolor{codegreen}{rgb}{0,0.6,0}
\definecolor{codegray}{rgb}{0.5,0.5,0.5}
\definecolor{codepurple}{rgb}{0.58,0,0.82}
\definecolor{backcolour}{rgb}{0.95,0.95,0.92}

% Configurar el estilo para el código
\lstdefinestyle{mystyle}{
  backgroundcolor=\color{backcolour},   
  commentstyle=\color{codegreen},
  keywordstyle=\color{magenta},
  numberstyle=\tiny\color{codegray},
  stringstyle=\color{codepurple},
  basicstyle=\ttfamily\footnotesize,
  breakatwhitespace=false,         
  breaklines=true,                 
  captionpos=b,                    
  keepspaces=true,                 
  numbers=left,                    
  numbersep=5pt,                  
  showspaces=false,                
  showstringspaces=false,
  showtabs=false,                  
  tabsize=2
}

%==============================================================================
% Cosas para la documentación LateX
% % Sangría
% \setlength{\parindent}{1em}Texto

% % Quitar sangría
% \noindent

% % Punto
% \CIRCLE \ \ \textbf{Texto} \emph{algo}
% \begin{itemize}
%   \item \textbf{Negrita:} Texto
%   \item \textbf{Negrita:} Texto
% \end{itemize}

% % Introducir código
% \begin{center}
%   \begin{BVerbatim}
%     ... Código
%   \end{BVerbatim}
% \end{center}

% Poner una imagen
% \begin{figure}[H]
%   \centering
%   \includegraphics[scale=0.55]{img/}
%   \caption{Exportación de la base de datos en formato sql}
%   \label{fig:exportación de la base de datos en formato sql}
% \end{figure}

% Poner dos imágenes
% \begin{figure}[H]
%   \begin{subfigure}{0.5\textwidth}
%     \centering
%     \includegraphics[scale=0.45]{img/}
%     \caption{Texto imagen 1}
%   \end{subfigure}%
%   \begin{subfigure}{0.5\textwidth}
%     \centering
%     \includegraphics[scale=0.45]{img/}
%     \caption{Texto imagen 2}
%   \end{subfigure}
%   \caption{Texto general}
% \end{figure}

% % Poner una tabla
% \begin{table}[H]
%   \centering
%   \begin{tabular}{|c|c|c|c|}
%     \hline
%     \textbf{Campo 1} & \textbf{Campo 2} & \textbf{Campo 3} & \textbf{Campo 4} \\ \hline
%     Texto & Texto & Texto & Texto \\ \hline
%     Texto & Texto & Texto & Texto \\ \hline
%     Texto & Texto & Texto & Texto \\ \hline
%     Texto & Texto & Texto & Texto \\ \hline
%   \end{tabular}
%   \caption{Nombre de la tabla}
%   \label{tab:nombre de la tabla}
% \end{table}

% % Poner codigo de un lenguaje a partir de un archivo
% \lstset{style=mystyle}
% The next code will be directly imported from a file
% \lstinputlisting[language=Python]{code.py}

% “Texto entre comillas dobles”

%==============================================================================

\begin{document}

% Portada del informe
\title{Trabajo Final - Cuaterniones}
\subtitle{Robotica Computacional}
\author{Cheuk Kelly Ng Pante (alu0101364544@ull.edu.es)}
\date{\today}

\maketitle

\pagestyle{empty} % Desactiva la numeración de página para el índice

% Índice
\tableofcontents

% Nueva página
\cleardoublepage

\pagestyle{plain} % Vuelve a activar la numeración de página
\setcounter{page}{1} % Reinicia el contador de página a 1

% Secciones del informe
% Capitulo 1
\chapter{Historia de los cuaterniones}
Los cuaterniones fueron creados por el matemático irlandés William Rowan Hamilton en 1843. Hamilton buscaba
formas de extender los números complejos a un número mayor de dimensiones. No pudo hacerlo para 3 dimensiones,
pero para 4 dimensiones obtuvo los cuaterniones. Hamilton estaba interesado en encontrar una forma de representar
las rotaciones en tres dimensiones mediante un solo número, ya que en ese momento no existía una forma de hacerlo.

En 1843, Hamilton tuvo una revelación mientras daba un paseo con su mujer. Según
cuenta la historia, Hamilton se detuvo de repente, sacó un lápiz y escribió los
símbolos i, j, k en una piedra cercana, y declaró que “aquí y ahora, con la ayuda de
Dios, el $i$, el $j$, el $k$ de la matemática acaban de recibir vida y movimiento”.
Inmediatamente, grabó esta expresión en el lateral del puente de Brougham, que estaba
muy cerca del lugar. Con esto, Hamilton había dado con la idea de los cuaterniones.

Sin embargo, a pesar del gran avance matemático que representaba, las ideas de
Hamilton no fueron recibidas con entusiasmo por la comunidad matemática de la
época. Muchos matemáticos estaban acostumbrados a trabajar con números reales
y complejos, y no podían entender cómo se podrían utilizar los cuaterniones en la
matemática. Además, el hecho de que los cuaterniones no cumplieran con la
conmutatividad generaba una gran confusión.

A pesar de esto, Hamilton no se dio por vencido y continuó trabajando en sus ideas
sobre los cuaterniones. En 1853, publicó un libro llamado "Lectures on
Quaternions", en el que describió en detalle las propiedades algebraicas de los
cuaterniones y cómo se podrían utilizar para representar rotaciones en el espacio
tridimensional.

Con el tiempo, los cuaterniones comenzaron a ser utilizados en áreas como la
graficación 3D, la robótica y la ingeniería aeroespacial. También se utilizaron en
algunos algoritmos de inteligencia artificial

\newpage

% Capitulo 2
\chapter{¿Que son los cuaterniones?}
Los cuaterniones son una extensión matemática de los números complejos y se utilizan en disciplinas relacionadas
con la representación y manipulación de orientaciones y rotaciones en el espacio tridimensional.

Los cuaterniones se utilizan para describir y manipular la orientación de objetos tridimensionales, como
robots, cámaras o herramientas. Cada cuaternión se compone de un componente escalar(imaginaria) y tres componentes
vectoriales(real). La forma general de un cuaternión se expresa como:
\begin{equation*}
  q = w + xi + yj + zk
\end{equation*}

Donde $w$ es el componente escalar y $x$, $y$ y $z$ son los componentes vectoriales. La unidad imaginaria
se representa por $i$, $j$ y $k$ y se definen como:
\begin{equation*}
  i^2 = j^2 = k^2 = ijk = -1
\end{equation*}

Los cuaterniones también se pueden representar mediante un vector de cuatro componentes (w, x, y, z) o una matriz $4x4$:
\begin{equation*}
  q = \begin{bmatrix}
    w & x & y & z
  \end{bmatrix}
\end{equation*}

\begin{equation*}
  q = \begin{bmatrix}
    w  & x  & y  & z  \\
    -x & w  & -z & y  \\
    -y & z  & w  & -x \\
    -z & -y & x  & w
  \end{bmatrix}
\end{equation*}

Los cuaterniones pueden ser de dos tipos:
\begin{itemize}
  \item \textbf{Cuaternión Unitario:} Es aquel cuyo valor absoluto es igual a 1.
  \item \textbf{Cuaternión No Unitario:} Es aquel que no cumple con el valor absoluto igual a 1.
\end{itemize}

\newpage

% Capitulo 3
\chapter{Aritmética básica de cuaterniones}
\begin{itemize}
  \item \textbf{Suma:} La suma de dos cuaterniones se hace como con los números complejos, es decir componente a componente.
        \begin{equation*}
          q_1 + q_2 = (w_1 + w_2) + (x_1 + x_2)i + (y_1 + y_2)j + (z_1 + z_2)k
        \end{equation*}
  \item \textbf{Producto:} El producto de dos cuaterniones se hace componente a componente.
        \begin{align*}
          q_1 \cdot q_2 & = (w_1 \cdot w_2 - x_1 \cdot x_2 - y_1 \cdot y_2 - z_1 \cdot z_2)        \\
                        & \quad + (w_1 \cdot x_2 + x_1 \cdot w_2 + y_1 \cdot z_2 - z_1 \cdot y_2)i \\
                        & \quad + (w_1 \cdot y_2 - x_1 \cdot z_2 + y_1 \cdot w_2 + z_1 \cdot x_2)j \\
                        & \quad + (w_1 \cdot z_2 + x_1 \cdot y_2 - y_1 \cdot x_2 + z_1 \cdot w_2)k
        \end{align*}
  \item \textbf{Conjugación:} La conjugación de un cuaternión se obtiene cambiando el signo de los componentes vectoriales.
        \subitem - El conjugado de un cuaternión $x = x_1 + x_2i +x_3j + x_4k$ está dado por $\overline{x} = x_1 - x_2i -x_3j - x_4k$
        \subitem - El valor absoluto de un cuaternión $x$ está dado por: $||x|| = \sqrt{x \cdot \overline{x}} = \sqrt{x_1^2 + x_2^2 + x_3^2 + x_4^2}$
  \item \textbf{Cociente:} El cociente entre cuaterniones se obtiene rápidamente a partir de la fórmula del inverso del cuaternión:
        \begin{equation*}
          q^{-1} = \frac{\overline{q}}{q \cdot \overline{q}}
        \end{equation*}
  \item \textbf{Exponenciación:} La exponenciación de un cuaternión; al igual que sucede con los números complejos, está relacionada con
        funciones trigonométricas. Dado un cuaternión escrito en forma canónica $q = a + bi +cj + dk$ su exponenciación resulta ser:
        \begin{equation*}
          e^q = e^{a+bi+cj+dk} = e^a \left( cos\sqrt{b^2+c^2+d^2} + \frac{sen \cdot \sqrt{b^2+c^2+d^2}}{\sqrt{b^2+c^2+d^2}} \cdot (bi + cj + dk) \right)
        \end{equation*}
\end{itemize}

\newpage

% Capitulo 4
\chapter{Aplicaciones}
Los cuaterniones tienen diversas aplicaciones que van desde la teoría de números, en donde pueden utilizarse par probar resultados como el teorema de los cuatro cuadrados,
hasta aplicaciones físicas dentro del electromagnetismo, la mecánica cuántica y la teoría de la relatividad, incluso en robots para realizar cirujias.

Los cuateriones se utilizan a menudo en gráficos por computadora, para representar la orientación de un objeto en el espacio tridimensional. También se utilizan en robótica,
para representar la orientación de un robot o de una herramienta. También se utilizan en la ingeniería aeroespacial, para representar la orientación de un avión o un satélite.

\section{Uso en lo videojuegos}
Los cuaterniones se utilizan en los videojuegos para representar la orientación de un objeto en el espacio tridimensional. Por ejemplo, se pueden utilizar para representar:
\begin{itemize}
  \item \textbf{Rotación de personajes y objetos: } Los cuaterniones se utilizan para representar la orientación de un personaje o un objeto en el espacio tridimensional.
        Por ejemplo, se pueden utilizar para representar la orientación de un personaje, un vehículo o una cámara.
  \item \textbf{Control de cámaras: } Los cuaterniones se utilizan para controlar la orientación de una cámara en un videojuego. Por ejemplo, se pueden utilizar para controlar
        la orientación de una cámara en un juego de disparos en primera persona.
\end{itemize}

Por lo que en el ámbito de los videojuegos es más facil utilizar cuaterniones que matrices, ya que permite crear animaciones más suaves.

Un programa para desarrollo de videojuegos y que utiliza cuaterniones es Unity.
Unity puede hacer uso de cuaterniones para representar la orientación o rotación de un objeto.

\section{Uso en los sistemas robotizados}
Los cuaterniones se utilizan en los sistemas robotizados para representar y controlar la orientación de un robot. Por ejemplo, se pueden utilizar para representar:
\begin{itemize}
  \item \textbf{Cinematica directa: } Con los cuaterniones en la cinematica directa se va calcular la posición y orientación del extremo final de un robot en función de las posiciones de sus
        articulaciones.
\end{itemize}

\section{Uso en la ingeniería aeroespacial}
Los cuaterniones se utilizan en la ingeniería aeroespacial para representar la orientación de un avión o un satélite. Por ejemplo, se pueden utilizar para representar:
\begin{itemize}
  \item \textbf{Control de orientación de la nave espacial: } Los cuaterniones se utilizan para describir la orientación de una nave espacial en el espacio. 
  \item \textbf{Navegación: } Los cuaterniones se utilizan para describir la posición y orientación de nave en el espacio en relación a un sistema de referencia.
\end{itemize}

Esto es especialmente útil para misiones automatizadas donde la nave espacial necesita mantener una orientación precisa sin la invervención humana.

\newpage

\chapter{Conclusiones}
Los cuaterniones son una extensión matemática de los números complejos y se utilizan en disciplinas relacionadas con la representación y manipulación de orientaciones y rotaciones en el espacio tridimensional.
Además, ofrecen una representación compacta y eficiente de la orientación, lo que facilita los cálculos en comparación con otras representaciones como las matrices de rotación y la composición de rotaciones 
mediante cuaterniones evita problemas como el bloqueo gimbal y proporciona una manera eficiente de manejar las rotaciones en el espacio tridimensional.


\end{document}